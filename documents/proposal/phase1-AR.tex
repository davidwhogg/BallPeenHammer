%%%%%%%%%%%%%%%%%%%%%%%%%%%%%%%%%%%%%%%%%%%%%%%%%%%%%%%%%%%%%%%%%%%%%%%%%%%
%
%    phase1-AR.tex  (use only for Archival Research and Theory proposals; 
%                     use phase1-GO.tex for General Observer and Snapshot
%                     proposals and phase1-DD.tex for GO/DD proposals).
%
%    HUBBLE SPACE TELESCOPE
%    PHASE I ARCHIVAL & THEORETICAL RESEARCH PROPOSAL TEMPLATE 
%    FOR CYCLE 21 (2013)
%
%    Version 1.0, December  1, 2012.
%
%    Guidelines and assistance
%    =========================
%     Cycle 21 Announcement Web Page:
%
%         http://www.stsci.edu/hst/proposing/docs/cycle21announce 
%
%    Please contact the STScI Help Desk if you need assistance with any
%    aspect of proposing for and using HST. Either send e-mail to
%    help@stsci.edu, or call 1-800-544-8125; from outside the United
%    States, call [1] 410-338-1082.
%
%%%%%%%%%%%%%%%%%%%%%%%%%%%%%%%%%%%%%%%%%%%%%%%%%%%%%%%%%%%%%%%%%%%%%%%%%%%

% The template begins here. Please do not modify the font size from 12 point.

\documentclass[12pt]{article}
\usepackage{phase1}

\newcommand{\project}[1]{\textsl{#1}}
\newcommand{\HST}{\project{HST}}
\newcommand{\WFC}{\project{WFC3}}
\newcommand{\MAST}{\project{MAST}}
\newcommand{\dd}{\mathrm{d}}

\begin{document}

%   1. SCIENTIFIC JUSTIFICATION
%       (see Section 9.1 of the Call for Proposals)
%
%
\justification          % Do not delete this command.
% Enter your scientific justification here.

\textbf{}


Precise calibration of the detector in an astronomical imaging camera
is not trivial---and it is only harder if it is in space.
Fundamentally, the sensitivity of the device must be calibrated with
incident photons; no artificial light source is available; even if an
artificial source \emph{were} available, it could not be designed to
illuminate the device exactly as does a star or other astronomical
source.

The imaging devices of \HST\ are calibrated by a number of
methods:
\\ $\bullet$ \project{laboratory calibration}: The instrument was illuminated
  in the laboratory pre-flight.  In principle this calibration can be
  done perfectly, but \textsl{(a)}~usually the calibration
  illumination does not illuminate the instrument exactly as does an
  in-flight star observation, and \textsl{(b)}~usually the light
  source does not have a completely appropriate SED.  The laboratory
  calibration happens pre-launch, so changes in the instrument through
  launch and age cannot be captured, nor can some aspects of in-launch
  environment (loading, temperature, pressure).
\\ $\bullet$ \project{standard stars}: On a regular schedule, well-understood
  standard stars are placed in the instrument focal plane to test the
  on-orbit throughput of the device.  These observations are key to
  instrument monitoring, but they don't calibrate the device at the
  pixel level; they provide only overall throughput measurements.
\\ $\bullet$ \project{grid test}: The throughput measurements are transferred
  out to the entire device by grid tests, in which standards or a star
  field are stepped over the instrument focal plane.  This brings the
  same astronomical source to many different focal-plane positions; it
  permits relative calibration of different parts of the detector.
  The grid, however, is not dense at the pixel level.  The grid test
  does not provide a pixel-to-pixel flat field; it only calibrates the
  flat on large scales.
\\ $\bullet$ \project{super-flat}: The only in-flight source of
  pixel-to-pixel sensitivity information are the photons detected in
  the sky---the blank parts of the imaging.  These photons can be
  combined (with masking of detected astronomical sources) into a
  pixel-level sensitivity map.  Unfortunately, this map is a
  sensitivity to the \emph{sky} not to a \emph{star}: The sky has a
  different SED than any star.  More importantly, the
  uniform-brightness sky illuminates the device differently than any
  star.  This problem is a bigger problem for open-structure
  ground-based telescopes than it is for \HST, but it isn't known
  to be negligible for \HST.

There is no method---not even any \emph{combination of methods}---that
can be used to all of \textsl{(1)}~test the calibration of the
detector in-flight, \textsl{(2)}~provide information on pixel-to-pixel
relative sensitivity (small scales), \textsl{(3)}~illuminate the
detector as stars do, and \textsl{(4)}~illuminate with the SED or
color of astronomical sources of interest.  It is possible that these
things don't matter---that every pixel of every \HST\ instrument is
properly calibrated---but it is close to impossible to know with the
calibration data available at present.  That is, if the flat
appropriate to stellar sources is different at fine (pixel-level)
angular scales from the flat appropriate to sky photons, that problem
would not appear strongly in current calibration data.

\textbf{Here we propose to calibrate \WFC\ at the pixel
  level using all of the science data availble in \MAST.}
Importantly, we will build this calibration from the astronomical
sources in the data, \emph{not} the blank-sky parts of the imaging.
By construction, the flats we produce will be built from sources with
astrophysically relevant SEDs.

In general there are two approaches for self-calibration of this kind.
The first---what we might call ``traditional'' self-calibration---is
built on the principle that if the \emph{same object} is observed at
\emph{different focal-plane positions} the inferences made ought to be
independent of focal-plane position.  This kind of self-calibration is
most effective when the observatory obeys calibration-oriented
observing strategies (CITE HOLMES), and those strategies involve
multiple observations for most sources.  The simplest kind of
self-calibration is the ``grid test'' mentioned above, but much more
sophisticated versions have been used to calibrate the \project{Sloan
  Digital Sky Survey} (CITE PADMANABHAN) and XXX (CITE).  Traditional
self-calibration (beyond the grid test) is not applicable to most
\HST\ instruments, because sources are rarely observed multiple times,
and when they are it is usually on one of a small number of
small-angle dithers.

In the second approach---what we might call ``probabilistic''
self-calibration---the fundamental principle is that no \emph{pixel}
will see a \emph{special set of astronomical objects}.  (Of course
this assumption is wrong in important ways, to which we will return
below.)  This kind of self-calibration is applicable in (almost) any
observing strategy, provided that the imager has been used to make
\emph{an enormous number of observations}.

In a probabilistic self-calibration, the idea is that the probability
distribution function (PDF) for image patches output by any exposure
in any region of the focal plane ought to match well the PDF in any
\emph{other} region of the focal plane.  If the calibration parameters
get set to bad values, the observed image patches will contain
imprints of the calibration errors, related to the location on the
focal plane from which they are taken.  The simplest version of
probabilistic self-calibration is the construction of the
``super-flat'' in the current \WFC\ calibration strategy; the
super-flat is constructed by forcing statistics of the pixel values to
be constant across the device.  The most extreme version of
probabilistic self-calibration would be to construct highly
informative prior PDFs for astronomical images and find the
calibration parameters at which the output images match best those
priors!  Here we propose to do something intermediate, much more
ambitious and informative than the super-flat, but tractable with
current \WFC\ data and computation.

%%%%%%%%%%%%%%%%%%%%%%%%%%%%%%%%%%%%%%%%%%%%%%%%%%%%%%%%%%%%%%%%%%%%%%%%%%%
%   2. ANALYSIS PLAN
%       (see Section 9.6 of the Call for Proposals)
%
%
\describearchival       % Do not delete this command.
% Enter your analysis plan here.

%%%%%%%%%%%%%%%%%%%%%%%%%%%%%%%%%%%%%%%%%%%%%%%%%%%%%%%%%%%%%%%%%%%%%%%%%%%

\textbf{We must include in `analysis plan', section 9.6 in Call -} 
i) Details of analysis and datasets to be analyzed.  ii) do APT
checklist iii) schedule for data delivery (is this in APT?) iv) what
docs, data products, and software will we deliver? v) how does this 
complement ongoing calib efforts?

The goal of our archival analysis is to perform probabilistic
self-calibration of WFC3 IR data.  Such aims can be very broad,
encompassing a range of calibration problems including static (or slowly 
evolving) issues (e.g., object and sky flat fields, ...) or transient
issues (e.g., persistence, `snow balls', ...).  Since our approach is
new and some-what exploratory, our analysis plan is to first focus on
achievable a baseline project, taking advantage of much of the current
knowledge of WFC3 (e.g., PSF model), and subsequently focusing on
enhanced-level deliverables.

\textbf{Our baseline goal is to produce a new, pixel-to-pixel
  flat field model for the WFC3 F110W and F160W filters.}\footnote{RF
  - Is there only one pixel-to-pixel filter derived on the ground? Or
  one for each filter?  Conceivably, if the former, we could use the existing
  low-freq. corrections to build up our own pixel-to-pixel filter, but
  using all the data from all the filters.} \footnote{RF - I think, if
  we need to produce a pixel-to-pixel flat \emph{per filter}, we need
  two filters so that Kalirai can do his MS test.} \footnote{RF - In
  Kalirai's paper the filters they beat on 47 Tuc with are F110W and
  F160W.}  Fundamentally, our method relies on the fact that stars
observed by HST are point sources, convolved by the PSF at the
location on the detector.  Beyond a simple normalization, therefore,
all stars ought to look similar if 1) the PSF (as a function of
time and position) is well understood, and 2) the pixel-to-pixel
calibration of the instrument is correct.  If, however, a there is a
pixel level error in the flat field, stars whos flux touches the
poorly calibrated pixel will look quite different when compared to the
flux distributions of all the other stars.  To illustrate, Figure
\ref{}...

More technically, we intend to build a pixel-level, generative model
for the data, under the best current PSF model for WFC3.  For a given
filter, we will take all images $M$ and divide them up into patches $D$
for which $N$ will have significant flux from an astronomical source.
The likelihood of a particular patch $i$ is 
\begin{eqnarray}
\mathcal{L}_i & = & p(D_i|P_{{\bf x},t},F_{{\bf x}})
\quad ,
\end{eqnarray}
where $P_{{\bf x},t}$ is the PSF over the spatial region of the
detector, at time $t$ (accounting for variations in time, such as
'breathing').  $F_{{\bf x}}$ is the flat field model, over the patch
$i$.  The total likelihood for all the data, therefore is just the
product of the likelihoods over all patches.  For our baseline
project, we will assume the best current PSF model for WFC3 (with
associated uncertainties) and only ask what pixel-to-pixel flat field
best explains the data.  Doing so, in a maximum likelihood sense,
amounts to optimizing the total likelihood of the patches with respect
to the flat field.

We plan to deliver (for our baseline) the pixel-to-pixel flat
fields (and software that produces them) for the F110W and F160W
filters.  We are choosing these particular filters since they are
amongst the most heavily used with WFC3 IR observations, having take
XX and YY images respectively.  In addition, we choose these filters
since they provide a set of tests with which we can assess the
fidelity of our flat fields.  Specifically, we will first examine the
consistency of the photometry of standard stars (gridded across 
the detector).  Subsequently, Jason Kalirai has volunteered to
reprocess WFC3 observations of 47 Tuc (see Kalirai et al. 2012), for
which we can examine the consistency of the Main Sequence of the
cluster.  In both cases, photometry will be performed on the
individual dithers, rather than the drizzled images (which may
introduce scatter, depending on the procedure).  By comparing the
results from our flat fields to those using standard \emph{calwfc3}
flats (as well as the flat fields themselves), \emph{we will produce
  the first verification of the pixel level flat field since the
  launch of WFC3}.  Beyond this zero-th result from our baseline
project, we expect we will learn new ways in which the calibration can
be improved, and may possibly provide an improved calibration of the
F110W and F160W filters.  Currently, the uncertainty in the flat
fields are about $0.5\%$, with a peak-to-peak variation of -1.5/1.6\%
(see ISR WFC3-2011-11).  While this is excellent, it will not deliver
the $<1\%$ photometry required by upcoming surveys (e.g., WFIRST,
Euclid).  We therefore view our exploratory archival calibration
project as a worthwhile step forward towards next-generation
calibration techniques.


%   3. MANAGEMENT PLAN
%       (see Section 9.7 of the Call for Proposals)
%
%  Provide a concise, but complete, management plan. This plan will be used
%  by the review panels to assess the likely scale of the proposed research
%  program. Proposers should include a schedule of the work required to
%  achieve the scientific goals of the program, a description of the roles of the
%  PI, CoIs, postdocs, and students who will perform the work, and a plan to
%  disseminate the results to the community.
%
\budgetnarrative       % Do not delete this command. CALLS the Management Plan header in the Style File (IGNORE the command name of budgetnarrative
% Enter your management plan here.

\textbf{Section 9.7 of Call says -} Provide a concise, but complete, 
management plan. This plan will be used by the review panels to assess 
the likely scale of the proposed research program. Proposers should 
include a schedule of the work required to achieve the scientific
goals of the program, a description of the roles of the PI, CoIs,
postdocs, and students who will perform the analysis, and a plan to 
disseminate the results to the community.  This document can be 8 
pages long (including figs/tables, past HST usage does not count),
this section can be 3 pages long.

Fadely will do this...

Hogg will do this...

summer student will do this...

We will travel to a few meetings...

%%%%%%%%%%%%%%%%%%%%%%%%%%%%%%%%%%%%%%%%%%%%%%%%%%%%%%%%%%%%%%%%%%%%%%%%%%%

%   4. PAST HST USAGE
%       (see Section 9.8 of the Call for Proposals)
%
%        List here the program numbers and data status for all accepted GO/AR/SNAP 
%        programs of the PI in at least the last four HST Cycles. Include a list of refereed publications 
%        resulting from these programs.       
%
%       Note that the description of past HST usage  DOES NOT count against the page limits of the proposal.
%
\pasthstusage  % Do not delete this command.

\textbf{This is needed, section 9.8 Call -} List here the program 
numbers and data status for all accepted GO/AR/SNAP programs of the 
PI in at least the last four HST Cycles. Include a list of refereed 
publications resulting from these programs. Calibration Proposals 
should describe what science will be enabled by the successful 
completion of the program, and how the currently supported core 
capabilities, their calibrations, and the existing pipeline or data
reduction software are insufficient to meet the requirements of this 
type of science.


None of the investigators has an approved \HST\ program in any of the
last four cycles.  That said, PI Hogg has been involved at a data
analysis and consulting level on the large \project{PHAT} project to
image the M31 disk.  This program has generated many refereed papers,
with co-authorship by Hogg only in...  ...Hogg has also been involved
in MARSHALL PROGRAMS?..


% List here the program numbers and data status for all accepted GO/AR/SNAP
% programs of the PI in at least the last four HST Cycles. Include a list of refereed
% publications resulting from these programs.

%%%%%%%%%%%%%%%%%%%%%%%%%%%%%%%%%%%%%%%%%%%%%%%%%%%%%%%%%%%%%%%%%%%%%%%%%%%

\end{document}          % End of proposal. Do not delete this line.
                        % Everything after this command is ignored.

