\documentclass[12pt]{article}

\newcommand{\documentname}{\textsl{Note}}
\newcommand{\foreign}[1]{\textsl{#1}}
\newcommand{\etal}{\foreign{et\,al.}}
\newcommand{\project}[1]{\textsl{#1}}

\begin{document}\sloppy\sloppypar

\noindent{\ttfamily This document is a draft.  It is Copyright 2012
  David W. Hogg, Ross Fadely, Rob Fergus, and others.  It is not ready
  for distribution.}

\begin{abstract}
In astronomical imaging projects, pixel-level calibration (bias, dark,
and flat) estimated via zero-length exposures and images of an
illuminated dome or twilight sky may not be optimal for application to
the measurements of greatest scientific interest.  Furthermore, in
most present-day experiments, many more photons are collected in the
scientific object data than in the calibration data.  For these
reasons, it makes sense to ask whether the full set of calibration
information could be derived from the science data alone.  In this
\documentname, we build very flexible models of astronomical imaging
based on mixtures of Gaussians (really mixtures of factor analyzers)
and use those models to test and improve pixel-level calibration
parameters.  We demonstrate with real data from the \project{Sloan
  Digital Sky Survey} and the \project{Hubble Space Telescope
  Wide-Field Camera 3} that we can accurately determine calibration
information using only science data.  Our self-calibration method is
computationally expensive, but has the advantages that (by
construction) it infers the parameters that are directly relevant to
the science data, that it captures the large amount of calibration
information latent in the full scientific data set, and that it
reduces or obviates calibration overheads.
\end{abstract}

...note to Hogg: abstract needs to mention ``no need to have repeat
imaging of the same part of the sky''...

\section{Introduction}

...on photometric self-calibration, cite Padmanabhan \etal~20xx, Holmes
\etal~2012.  Note limitation that these methods work at catalog level,
don't calibrate below the resolution of the photometric methodology,
and aren't sensitive to additive problems.

...Notes about flexible models, maybe even citing a previous paper by
Fadely?  It is not clear to Hogg whether these should be separate
papers or the same paper.

\section{Image patch model}

...Hogg proposes that we be very consistent with indexing.  I propose
that we have $N$ $d$-pixel (say $d=49$ or $81$) image patches $n$ on
which we train the model.  The model will have $K$ Gaussian components
$k$, each of which is a mixture of factor analyzers with $M\ll d$
factors $m$.  The imaging detector will have $J$ independent pixels
(or, in the case of \project{SDSS}, pixel columns) $j$.  The detector
has been used to take $H$ exposures $h$, each of which has (we hope) a
different exposure time $t_h$ and has seen (we hope) a different
astronomical scene.


...We need an astronomer-friendly explanation of mixture of factor
analyzers.  Hogg is at your service.

\section{Calibration parameter inference}

...something like: For every pixel position $j$ in the detector, there
is a zero level $z_j$, a dark current rate $d_j$, and a sensitivity
(or effective area) $a_j$.  Detector pixel $j$ also has $7\times
7-1=48$ (fewer if it is an edge pixel) neighbor pixels $\ell$.  In the
raw data, in exposure $h$...

\section{Data and results}

\section{Discussion}

\end{document}
