\documentclass[12pt]{article}

\newcommand{\documentname}{\textsl{Note}}
\newcommand{\project}[1]{\textsl{#1}}

\begin{document}\sloppy\sloppypar

\noindent{\ttfamily This document is a draft.  It is Copyright 2012
  David W. Hogg, Ross Fadely, Rob Fergus, and others.  It is not ready
  for distribution.}

\begin{abstract}
In astronomical imaging projects, pixel-level calibration (bias, dark,
and flat) estimated via zero-length exposures and images of an
illuminated dome or twilight sky may not be optimal for application to
the measurements of greatest scientific interest.  Furthermore, in
most present-day experiments, many more photons are collected in the
scientific object data than in the calibration data.  For these
reasons, it makes sense to ask whether the full set of calibration
information could be derived from the science data alone.  In this
\documentname, we build very flexible models of astronomical imaging
based on mixtures of Gaussians (really mixtures of factor analyzers)
and use those models to test and improve pixel-level calibration
parameters.  We demonstrate with real data from the \project{Sloan
  Digital Sky Survey} and the \project{Hubble Space Telescope
  Wide-Field Camera 3} that we can accurately determine calibration
information using only science data.  Our self-calibration method is
computationally expensive, but has the advantages that (by
construction) it infers the parameters that are directly relevant to
the science data, that it captures the large amount of calibration information latent
in the full scientific data set, and that it reduces or obviates
calibration overheads.
\end{abstract}



\end{document}
