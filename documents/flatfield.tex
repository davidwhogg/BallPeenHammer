\documentclass[12pt,preprint,pdftex]{aastex}

\newcommand{\project}[1]{\textsl{#1}}

\begin{document}

\title{Don't waste your time taking flats!}
\author{
  Ross~Fadely\altaffilmark{1,2},
  David~W.~Hogg\altaffilmark{1,3},
  Daniel~Foreman-Mackey\altaffilmark{1}
  Dilip~Krishnan\altaffilmark{4}, and
  Rob~Fergus\altaffilmark{4}
}
\altaffiltext{1}{Center for Cosmology and Particle Physics, Department of Physics, New York University, 4 Washington Place, New York, NY 10003, USA}
\altaffiltext{2}{To whom correspondence should be addressed.}
\altaffiltext{3}{Max-Planck-Institut f\"ur Astronomie, K\"onigstuhl 17, D-69117 Heidelberg, Germany}
\altaffiltext{4}{Department of Computer Science, New York University}

\begin{abstract}
Typical astronomical imaging devices measure the intensity field but
only after multiplication by a non-uniform pixel sensitivity (flat
field) and offset by a non-uniform zero or dark image (plus noise and
sometimes nonlinearities).  Measurements of the flat-field and dark
are usually made with calibration images taken during an observing
campaign.  When it comes to the flat-field, none of the calibration
images that can be easily taken illuminate the detector in precisely
the same way as an astronomical source would.  Here we show that the
flat-field and dark properties of a detector can be inferred from the
science data alone, without the use of any calibration data
whatsoever, provided that the science data set is large and diverse
enough in its properties.  The method operates by building a
data-driven ($K$-nearest-neighbors) model of small image patches and
then adjusting the dark and flat until the statistical properties of
small pixel patches become uniform across the device.  We demonstrate
our methods on \project{Sloan Digital Sky Survey} data.
Computationally, calibration of a device using the science data alone
is expensive, but it has the great advantage that the device is being
calibrated with correct illumination, by construction.
\end{abstract}

\noindent
\textsl{This document is a DRAFT dated 2012-11-21, not ready for distribution.}

\section{Introduction}

Why both twilight and dome flats suck, twilight in particular.

Superflats are the same as twis.  SDSS scary story.

Stars illuminate the detector precisely as stars do!

\section{Method}

In short, the algorithm is:
\begin{enumerate}
\item Make a first guess at the flat and dark.  Each of these is an image in device pixels $i$; that is, there is one dark value $d_i$ and one flat value $f_i$ per device pixel $i$.
\item Calibrate the science imaging data; that is, subtract the dark (times the exposure time) and divide by the flat.\label{step:calibrate}
\item Cut the calibrated imaging data into small $D$-pixel patches $n$.
\item Begin loop over device pixels $i$.
\item Begin loop over patches $n$ that have device pixel $i$ in the ``central pixel'' location.
\item Find the $K$ nearest neighbor patches $k$ to patch $n$, where by ``nearest neighbor'' we mean nearest in the $D-1$ dimensional space excluding the central pixel, and possibly also subject to optimal rescaling and offsetting, and definitely also excluding patches that touch device pixel $i$.
\item Begin loop over neighbor patches $k$.
\item Find the optimal shift and scale in intensity that makes the $D-1$ non-central pixels in patch $k$ look most like the $D-1$ non-central pixels in patch $n$.  After that optimal shift and scale, the central pixel of patch $k$ has value $I_{nk}$.
\item End loop over neighbor patches $k$.  Compute the mean $m_n$ and variance $\sigma^2_n$ of the values $I_{nk}$ over the $K$ neighbors $k$.
\item Compute the probability of the central pixel of patch $n$ under the single Gaussian with mean $m_n$ and variance $\sigma^2_n$ as a function of adjustments $(\Delta d_i, \Delta f_i)$ to the dark and flat at device pixel $i$.  This is the patch-$n$ likelihood function for adjustments to pixel $i$.
\item End loop over patches $n$.  Compute the product of the patch-$n$ likelihoods (or sum of log likelihoods).  Optimize this total likelihood for pixel $i$ to obtain adjustments to the dark and flat at pixel $i$.
\item Update the dark and flat at pixel $i$.
\item End loop over device pixels $i$.  Renormalize flat so the mean flat is unity.
\item Iterate by returning to step~\ref{step:calibrate}.
\end{enumerate}

\end{document}

