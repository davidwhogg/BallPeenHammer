\documentclass[12pt,preprint,pdftex]{aastex}

\newcommand{\documentname}{\textsl{Note}}
\newcommand{\equationname}{equation}
\newcommand{\foreign}[1]{\textsl{#1}}
\newcommand{\etal}{\foreign{et\,al.}}
\newcommand{\project}[1]{\textsl{#1}}
\newcommand{\given}{\,|\,}
\newcommand{\setofall}[1]{\left\{{#1}\right\}}
\newcommand{\dd}{\mathrm{d}}

\begin{document}\sloppy\sloppypar

\noindent{\ttfamily This document is a draft.  It is Copyright 2012
  David W. Hogg, Ross Fadely, Rob Fergus, and others.  It is not ready
  for distribution.}


\section{Calibration using 1x2 patches of pixels}

We can up images into $1\times2$ pixel patches, which has the advantage of easy 
visualization and computation.  For the (fake but realistic) astronomical images we 
have been analyzing, the distribution of pixel values is characterized by something 
like the distribution show in Figure \ref{fig:fig1}.  There is a locus of pixels centered on 
$(0,0)$, these are the parts of the image with no sources.  The fan of pixels with larger 
(positive) pixel values are two pixel patches which hit an astronomical source.  

In Figure \ref{fig:fig1}, the `True' distribution of values are shown, along with the distribution 
for the same set of images but with a flatfield error of 0.8 (1.2), alternating in even (odd) 
columns, respectively.  The point to notice is that the variance perpendicular to the `$y=x$' 
axis is larger for the data with the flatfield error.

\begin{figure}
\centering
 \includegraphics[clip=true, trim=0cm 0cm 0.0cm 0.cm,width=12cm]{/Users/rossfadely/BallPeenHammer/plots/pixelvalues_2d.png}
\caption{Distribution of pixel values for $(1\time2)$ pixel patches, extracted from simulated images.  The corrupted data has a flatfield error of $\pm20\%$, alternating by column.  Note the larger variance perpendicular to the $y=x$ axis for the corrupt data. This is the distribution from 10 random images.  The thick red line shows the variance cut used below.}
\label{fig:fig1}
\end{figure}

A way to quantify the significance of the variance is the asymmetry\footnote{Hogg claims jargon from the world of particle physics.}:

\begin{eqnarray}
A &=& \frac{x-y}{x+y}\\
f(\hat{x_k}) &=& \frac{1}{N}\sum^{N}_i\left[\prod_j^{J} h_j^{-1} K\left(\frac{\hat{x_{kj}}-x_{ij}}{h_j}\right)\right]
\end{eqnarray}

\noindent where $x, \,y$ are the first and second pixel values in the patch.  In Figure \ref{fig:fig2} the asymmetry is plotted for the patches in Figure \ref{fig:fig2}, after cutting on the variance (actually total intensity) of the patches (red line, Figure \ref{fig:fig1}.  It is clear that (in this exaggerated case) that the variance of the corrupted distribution is larger than the true distribution.  The same version of Figure \ref{fig:fig2} is shown in Figure \ref{fig:fig3}, for $10\times$ the number of images.

\begin{figure}
\centering
 \includegraphics[clip=true, trim=0cm 0cm 0.0cm 0.cm,width=12cm]{/Users/rossfadely/BallPeenHammer/plots/asym.png}
\caption{Distribution of asymmetry values for the patches shown in Figure \ref{fig:fig1}, but only for high variance sources (red line in the above figure).  The variance for the corrupted data is larger than that of the true distibution.}
\label{fig:fig2}
\end{figure}

\begin{figure}
\centering
 \includegraphics[clip=true, trim=0cm 0cm 0.0cm 0.cm,width=12cm]{/Users/rossfadely/BallPeenHammer/plots/asym_100.png}
\caption{Same as Figure \ref{fig:fig2}, but for $10x$ the number of images.}
\label{fig:fig3}
\end{figure}


\end{document}

