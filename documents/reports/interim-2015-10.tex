\documentclass[12pt,letterpaper]{article}
\setlength{\parskip}{0.0ex}
\setlength{\parindent}{0.2in}
\renewenvironment{itemize}{\begin{list}{$\bullet$}{%
  \setlength{\topsep}{0.5ex}%
  \setlength{\parsep}{0.0ex}%
  \setlength{\partopsep}{0.0ex}%
  \setlength{\itemsep}{0.5ex}%
  \setlength{\leftmargin}{1.5\parindent}}}{\end{list}}
\newcommand{\course}[1]{\textsl{#1}}

\begin{document}
\thispagestyle{empty}
\noindent
\begin{tabular}{@{\textbf}ll}
To:      & Grants Administration Office, STScI \\[1ex]
From:    & Ross Fadely \& David W. Hogg, New York University \\[1ex]
Subject: & HST-AR-13250.002-A interim report \\[1ex]
Date:    & 2015-10-31 \\[1ex]
\end{tabular}
\bigskip

\noindent
This memo constitutes an interim report for STScI Archival Research
Grant HST-AR-13250.002-A ``Probabilistic Self-Calibration of the WFC3
IR Channel'', Fadely, PI.

The project aims to deliver new calibration vectors for the WFC3 IR
Channel, by building a model of all the science imaging in the HST
Archive.  The model is a model of the point-spread function of the
instrument, and a model of every image as a set of point sources,
convolved with the point-spread function, and multiplied by the
flat-field.  The project is currently drafting a paper about
data-driven point-spread-function estimation, and once this is
completed, will infer the flat-field.  In some more detail:
\begin{itemize}
\item
We have developed a model for the pixel-by-pixel flat field for the
WFC3 IR channel.  For data, we use all the FLT images collected in the
F160W filter.  In short, the model optimizes (or samples) the log
likelihood of the central 5x5 pixel patches of stars, using the PSF
model from Jay Anderson (STScI).  Each patch is modeled as an amplitude times
the PSF (plus a constant background amplitude), times the flat field
model for the pixels involved.
\item
As a test of the validity of this procedure, we generated realistic
(but fake) data and examined how well we can recover the 'true' values
for the amplitudes and flat field values.  We find that the optimized
likelihood yields parameter values very close to the input, and that
(while slow) blocked Gibbs sampling produces posteriors consistent
with the input values.
\item
For the real HST data, we find that this model produces log likelihood
values for held out (validation) data which are higher than models
without a pixel level flat field.  For stars with repeated
observations (at different dithers), the model produces amplitudes
(fluxes) for the stars whose variance is small than models without a
pixel level flat field.  In other words, the photometry appears more
consistent (as expected from fake data tests).
\item
The project is in the finishing stages.  Details about accounting for
misidentified stars (or bad data) are being finalized.  A document and
figures are being prepared for publication.
\end{itemize}

During the granting period, the grant directly supported part of Ross
Fadely's salary, and indirectly supported graduate student
M.J. Vakili, who is using the data and code and working on the paper
on the point-spread function.

\end{document}
